This paper proves that exogenous empirical-evidence equilibria (xEEEs) in perfect-monitoring repeated games induce correlated equilibria of the associated one-shot game.
An empirical-evidence equilibrium (EEE) is a solution concept for stochastic games.
At equilibrium, agents’ strategies are optimal with respect to models of their opponents.
These models satisfy a consistency condition with respect to the actual behavior of the opponents.
As such, EEEs replace the full-rationality requirement of Nash equilibria by a consistency-based bounded-rationality one.
In this paper, the framework of empirical evidence is summarized, with an emphasis on perfect-monitoring repeated games.
A less constraining notion of consistency is introduced.
The fact that an xEEE in a perfect-monitoring repeated game induces a correlated equilibrium on the underlying one-shot game is proven.
This result and the new notion of consistency are illustrated on the hawk-dove game.
Finally, a method to build specific correlated equilibria from xEEEs is derived.
